\tikzset{
  highlight/.style={draw=red, thick, circle, inner sep=2pt}
}

\begin{tikzpicture}
    \begin{feynman}
      %
      % -- Define the incoming quarks on the left
      \vertex (qbar) at (-2, 1.5) {\(q\)};
      \vertex (q)    at (-2,-1.5) {\(\bar{q}\)};
      
      % -- Central vertex where q qbar' -> W_R
      \vertex [dot] (wr) at (0,0) {};
      
      % -- First decay: W_R -> N_\ell + l
      \vertex [dot] (Nl) at (2.5,0) {};
      \vertex (l1)  at (4,-1.5) [highlight] {\(\ell\)};
      \vertex (N) at (4, 1.5) {};
      
      % -- Second decay: N_\ell -> l + W_R
      \vertex [dot] (wrl) at (4, 1.5) {};
      \vertex (l2) at (5.5,  3) [highlight] {\(\ell\)};
      \vertex (wr2) at (5.3,  0.5) {};

      %    Then W_R -> 2 jets
      \vertex [dot] (jj) at (5.3, 0.5) {};
      \vertex (jet1) at (6.5,  1.7) [highlight] {jet};
      \vertex (jet2) at (6.5, -0.5) [highlight] {jet};
      
      % -- Draw the diagram
      \diagram*{
        % Incoming quarks
        (qbar) -- [fermion] (wr) -- [fermion] (q),
        
        % First decay of W_R
        (wr) -- [boson, edge label=\(W_R\)] (Nl),
        (l1) -- [fermion] (Nl),
        (Nl) -- [fermion, edge label=\(N_\ell\)] (wrl),
        
        % Decay of heavy neutrino N_l
        (wrl) -- [fermion] (l2),
        (wrl) -- [boson, edge label=\(W_R^{*}\)] (jj),

        % Decay of second W_R into jets
        (jet1) -- [fermion] (jj) -- [fermion] (jet2),
      };
    \end{feynman}
  \end{tikzpicture}